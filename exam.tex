\documentclass[12pt,answers]{exam}

\usepackage{ctex} %%中文字体设置

%%----------------------页面大小设置--------------------------
\usepackage{geometry}
\geometry{%
	paperwidth=18.5cm,
	paperheight=26cm,
	top=2.5cm,
	bottom=2.2cm,
	left=2cm,
	right=2cm,
}

%%----------------------表格排版----------------------------
\usepackage{array}


%%---------------------线框水印------------------------------
%%----------------------------------------------------------
%%----------------------------------------------------------
\usepackage{tikz}
\usepackage{eso-pic}
%%---------------------首页线框------------------------------

\newcommand{\thiswatermark}[3]{\AddToShipoutPictureBG*{ 
		\parbox[b][\paperheight]{\paperwidth}{
			\begin{tikzpicture}
				\node (x) at (0cm,0cm) {};%定位用
				\draw[dashed,very thick](1.5cm,20.2cm) rectangle (16.5cm,23.5cm);
				\draw[thick] (1.5cm,1.7cm) rectangle (16.45cm,14.3cm);
				\draw (1.5cm,13.6cm) -- (16.45cm,13.6cm);
				\node (a) at (3cm,13.94cm) {考\ 生\ 班\ 级};
				\node (b) at (8cm,13.94cm) {学\ \ 号};
				\node (c) at (12.8cm,13.94cm) {姓\ \ 名};
				\draw (4.35cm,13.6cm)--(4.35cm,14.3cm);
				\draw (6.75cm,13.6cm)--(6.75cm,14.3cm);
				\draw (9.1cm,13.6cm)--(9.1cm,14.3cm);
				\draw (11.6cm,13.6cm)--(11.6cm,14.3cm);
				\draw (14cm,13.6cm)--(14cm,14.3cm);
			\end{tikzpicture}

		}
		}
		}

%%-----------------------其余页线框--------------------------

\newcommand{\watermark}[3]{\AddToShipoutPictureBG{ 
		\parbox[b][\paperheight]{\paperwidth}{
			\begin{tikzpicture}
				\node (x) at (0cm,0cm) {};%定位用
				\draw[thick] (1.5cm,1.7cm) rectangle (16.45cm,23.89cm);
			\end{tikzpicture}
			
		}
	}
}
%%----------------------------------------------------------
%%----------------------------------------------------------
%%----------------------------------------------------------



%%--------------------页眉页脚------------------------------
%%----------------------------------------------------------
%%----------------------------------------------------------
\pagestyle{headandfoot} %% 源自exam宏包参数
\runningheadrule
\firstpageheader{}{}{}
\runningheader{}{西北工业大学命题专用纸}{}
\firstpagefooter{\fontsize{10pt}{10pt}\selectfont \vspace{-0.8cm}注:1. 命题纸上一般不留答题位置,试题请用小四、宋体打印且不出框。\\ \hspace{0.7cm}2. 命题教师和审题教师姓名应在试卷存档时填写。}{}{\fontsize{10pt}{10pt}\selectfont 共\numpages 页,第\thepage 页}
\runningfooter{\fontsize{10pt}{10pt}\selectfont \vspace{-0.65cm}教务处印制}{}{\fontsize{10pt}{10pt}\selectfont \vspace{-0.65cm} 共\numpages 页,第\thepage 页}
%%----------------------------------------------------------
%%----------------------------------------------------------
%%----------------------------------------------------------



%%---------------------折线不换行---------------------------
\usepackage{ulem}
\usepackage{CJKulem}
\ExplSyntaxOn
\RenewDocumentCommand{\fillin}{m}{\tl_map_inline:nn{#1}{\uline{##1}\hspace{0cm}}}
\ExplSyntaxOff%%不换行问题并未完全解决


%%-----------------------正文开始---------------------------
%%----------------------------------------------------------
%%----------------------------------------------------------
\begin{document}
	
%%-----------------------封面内容---------------------------
	\thiswatermark{}{}{} %调用首页水印背景
	\clearpage

	\begin{center}
		\fontsize{12pt}{12pt}\selectfont 诚信保证
	\end{center}
	
	\vspace{-0.5cm}
	\par \fontsize{12pt}{18pt}\selectfont 本人知晓我校考场规则和违纪处分条例的有关规定,保证遵守考场规则,诚实做人。\hspace{8.5cm} 本人签字:\underline{\makebox[2cm]{}}
	
	\noindent 编号:\underline{\makebox[1.8cm]{}}
	
	\begin{center}  
		\fontsize{18pt}{26pt}\selectfont 西北工业大学考试试题(卷)\\
		
		\fontsize{12pt}{18pt}\selectfont 20\ \ -20 \ \ 学年第\ \ \ 学期
	\end{center}
	\vspace{-0.5cm}
	\noindent 开课学院\underline{\makebox[3.4cm]{}} \hspace{0.9cm} 课程\underline{\makebox[3.4cm]{}} \hspace{0.9cm}  学时\underline{\makebox[2cm]{}}
	
	\noindent 考试日期\underline{\makebox[2cm]{}} \hspace{0.35cm} 考试时间 \underline{\makebox[2cm]{}}小时  \hspace{0.35cm} 考试形式(开/闭)(A/B)卷
	
	\begin{table}[h]
		\centering
		\begin{tabular}{|>{\centering\hspace{0pt}}m{0.1\linewidth}|>{\centering\hspace{0pt}}m{0.075\linewidth}|>{\centering\hspace{0pt}}m{0.075\linewidth}|>{\centering\hspace{0pt}}m{0.075\linewidth}|>{\centering\hspace{0pt}}m{0.075\linewidth}|>{\centering\hspace{0pt}}m{0.075\linewidth}|>{\centering\hspace{0pt}}m{0.075\linewidth}|>{\centering\hspace{0pt}}m{0.075\linewidth}|>{\centering\arraybackslash\hspace{0pt}}m{0.1\linewidth}|} 
			\hline
			题号 & 一 & 二 & 三 & 四 & 五 & 六 & 七 & 总分  \\ 
			\hline
			得分 &   &   &   &   &   &   &   &     \\
			\hline
		\end{tabular}
	\end{table} 


%%----------------------试题正文开始------------------------
%%----------------------------------------------------------
	
	\begin{questions}
		\question 同非金属相比,金属的主要特性是\fillin[导电性]
		\question 
		\question 
		\question 
		\question 
		\question 
		\question 
		
		
%%--------------------此处正文前后手动调节--------------------
%%----------------------------------------------------------
	\newpage
	\watermark{}{}{}	
		
		\question 
		\question 
		\question 
		\question
		\question 
		\question 
		\question 
		\question
		\question 
		\question 
		\question 
		\question 
		\question 
		\question 
		\question 
		\question 
		\question 
		\question 
		\question 
		\question 
		\question 
		\question 
		\question 
		\question 
		\question 
	\end{questions}




\end{document}